% !TEX root = ../main.tex
\chapter{Related Work}
\label{chapter:related_work}

\section{The theory of dependent types}

\section{Dependently typed programming languages}

Given the correspondence between dependent type systems and propositional logic.
The first programming languages with dependent types were proof assistants, such
as Coq and Agda, \cite{agda} which can be used to proof theorems automatically. 
However, such languages are not used outside the academic community nor to write 
user or system oriented applications.

Haskell, being both a general purpose programming language and the de facto
language to explore and study new type systems in the academy, has provided the
foundation to study several formulations of dependent types in the form of new
languages, such as Agda, Idris \cite{idris} and Cayenne, \cite{cayenne} or by 
adding dependent types directly to the language. This last part is particularly
relevant to this work, given that Rust's nor Haskell's type system and compiler 
were written to have dependent types.

C++ is the language that is most often compared to Rust because both languages
were designed to write performance oriented applications. Templates, which are
C++ mechanism to achieve parametric polymorphism, allow not only types as
parameters but also constant values. \cite{templates} This idea is pretty 
similar to the one being implemented for Rust in the RFC 2000. Even then, 
manual memory allocation makes C++ a language far from being memory safe. 
Where as Rust is a completely memory safe language.    

\subsection{Dependent Haskell}

\subsection{Templates in C++}

\section{The state of Rust}
\subsection{Rust's type system}
\subsection{Rust RFC 2000}
\subsection{The typenum crate}
