% !TEX root = ../main.tex
\chapter{Motivation}
One of the main objectives of Rust is to reduce the number of trade-offs for the programmer. For example, fast code should not be unsafe, and abstractions should not have significant performance costs. Rust accomplishes this by having a compiler that is able to reason more deeply about the code than usual. Rust can reason about the lifetime of each variable to avoid dangling pointers, or about mutability to avoid data races. 

However, some improvements could be done in the interaction of the language with constant values. Not necessarily from a performance perspective, but also from an ergonomic perspective. This chapter will explain some aspects of the compilation and reasoning processes that could be improved. Chapter \ref{chapter:proposed_solution} will address the solution for the problems posit hereinafter.

\label{chapter:motivation}
\section{Trait implementations for arrays}
In Rust, the programmer has the capability to store data in the heap and the stack. On the one hand, stack allocated values must have an static size (it must be known at compilation). On the other hand, heap allocated values can have dynamic size, but they can only be accessed using references. The differences between vectors and arrays are a perfect example of this trade-off: Vectors, being heap allocated, can grow or reduce in size during execution, but having nested vectors causes performance issues, because access must be done using nested references. Arrays, being stack allocated, have no performance issues when nesting them, but their size must be fixed during execution.

These limitations create some ergonomic problems with arrays: Two arrays with different sizes have different types, even if both store values of the same type. Thus, trait implementations for array types must be done manually for each possible size. Because of this, the standard library only implements traits for array types up to a size of 32, even though arrays are a primitive type of the language. In certain cases vectors are used instead of arrays. Making the code unsuitable for applications running on embedded devices (which may not allow heap allocation) or applications needing multidimensional vectors.

An example of this limitation is shown in \ref{lst:trait_array}, where the implementation of the trait \texttt{Volatile} cannot be easily extended to other array sizes, even when such implementation does not use the size of the array. 

\begin{listing}
	\begin{minted}{rust} 
    const N: usize = 3;

    trait Volatile {
        fn explode(&self);
    }

    impl<T> Volatile for [T; N] {
        fn explode(&self) {
            for _ in self {
                println!("Boom!")
            }
        }
    }

    fn main() {
        [0i32, 1, 2].explode(); 
        [0i32, 1].explode(); // stderr: no method named `explode` 
                             // found for type `[i32; 2]` in the
                             // current scope
    }
	\end{minted}
    \caption{Even though \texttt{Volatile} is implemented for \texttt{[T; 3]}, it is not for \texttt{[T;2]}.}
  \label{lst:trait_array}
\end{listing}

\section{Static control flow and optimizations}
Rust's compiler is capable of doing static control flow to improve performance execution. If a control flow statement can be resolved during compilation, the compiler will remove the unused expressions of the statement. For example, if the condition of an if-else statement can be resolved during compilation, the compiler will replace the statement with one just including the matching arm of the statement.

An small illustration of this can be seen in \ref{lst:static_control_flow}, where the array size \texttt{N} is a constant value known during compilation. Given that the expression  \texttt{N > 0} can be evaluated during compilation, the compiler will optimize the function, resulting in code similar to the code shown in \ref{lst:optimized}. However, doing such optimizations increases compilation time. 

This can be seen as a trade-off between compilation time and code generality. If the programmer chooses to use \ref{lst:static_control_flow}, the code will work even if \texttt{N} needs to be changed afterwards, if the programmer chooses the code in \ref{lst:optimized}, the code will compile faster. But the programmer can not achieve both at the same time.

\begin{listing}
    \begin{minted}{rust}
    const N: usize = 0;

    fn head<T>(array: [T; N]) -> Option<T> {
        if N > 0 {
            Some(array[0])
        } else {
            None
        }
    }
    \end{minted}
    \caption{This function must be optimized to improve performance.}
    \label{lst:static_control_flow}
\end{listing}

\begin{listing}
    \begin{minted}{rust}
    fn head<T>(array: [T; 0]) -> Option<T> {
        None
    }
    \end{minted}
    \caption{This function should be equivalent to the one in \ref{lst:static_control_flow}.}
    \label{lst:optimized}
\end{listing}
