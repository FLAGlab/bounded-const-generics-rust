% !TEX root = ../main.tex
\chapter{Validation}
\label{chap:validation}

Having discussed the design of our solution, this chapter addresses the
advantages and disadvantages of using this new kind of generics on an small
linear algebra module. This particular topic was chosen, given that numeric
computing makes heavy usage of statically sized linear data structures to
represent matrices and vectors.

Using the Rust's \inrust{array} primitive type would be an unfair comparison,
given that in its current state, Rust is not able to abstract the size of the
arrays in a satisfactory manner. Instead we use the standard library
\inrust{Vec}, which is dinamically sized but it can be wrapped to restrict its
size.

Listings \ref{lst:vector_vanilla} and \ref{lst:vector_const_gen} contain two
implementations of a \inrust{Vector} structure with and without using generics
over constant values. Each implementation has three methods:

\begin{itemize}
    \item The \inrust{dot} method: Compute the dot product between the current
        vector and another vector given as parameter, it is required that both
        vectors have the same number of dimensions.  
    \item The \inrust{append} method: Consume the current vector and return a
        new vector which has an additional dimension with an element given as
        paramter.  
    \item The \inrust{split} method: Consume the current vector and return two
        vectors which contain the two halves of the current vector.  
\end{itemize}

The first noticeable difference between the two implementations is the usage of
unsafe code in \Fref{lst:vector_const_gen}. The \inrust{from_vec} method is
declared as unsafe given that there is no way to verify during compilation that
the size of the vector will correspond with its constant type parameter,
meaning that is possible to create a \inrust{Vector<N>} variable from a
\inrust{Vec} with a size different from \inrust{N}. Given its unsafety, this
method is not public.

\begin{listing}[H]
    \begin{minted}{rust}
    pub struct Vector { inner: Vec<f32> }

    impl Vector {
        pub fn dot(&self, other: &Vector) -> Option<f32> {
            if self.inner.len() == other.inner.len() {
                let mut res = 0.0;
                for i in 0..self.inner.len() {
                    res += self.inner[i] * other.inner[i];
                }
                Some(res)
            } else {
                None
            }
        }

        pub fn append(mut self, value: f32) -> Vector {
            self.inner.push(value);
            self
        }

        pub fn split(self) -> (Vector, Vector) {
            let (a, b) = self.inner.split_at(self.inner.len() / 2);
            (Vector { inner: a.to_vec() }, 
             Vector { inner: b.to_vec() })
        }
    }
    \end{minted}
    \caption{A vector implementation without generics over constants}
  \label{lst:vector_vanilla}
\end{listing}

Another difference is the return type in the \inrust{dot} method. The
\Fref{lst:vector_vanilla} implementation return type is \inrust{Option<f32>},
this is because it is necessary to verify that both vectors have the same size
during runtime. In contrast, the \Fref{lst:vector_const_gen} implementation
does not require this verification inside the method because it is done during
compilation, the \inrust{other} vector must have size \inrust{N} and, as a
consequence, calling this method with a vector of a different size would result
on a compile time error.

The differences on the \inrust{append} method reside in the return type. On
\Fref{lst:vector_const_gen} the size of the return vector is increased by one
on its type parameter. However, in order to satisfy this, is necessary to
transmute the current vector (which has type \inrust{Vector<N>}) to change its
type to \inrust{Vector<N+1>}. This change cannot be done safely given that its
overriding the type system of the language. In contrast, the implementation on
\Fref{lst:vector_vanilla} does not require this use of unsafe code, because the
size of the vector is not encoded in its type.

\begin{listing}[H]
    \begin{minted}{rust}
    pub struct Vector<const N: usize> { inner: Vec<f32> }

    impl<const N: usize> Vector<N> {
        unsafe fn from_vec(inner: Vec<f32>) -> Vector<N> {
            Vector { inner }
        }

        pub fn dot(&self, other: &Vector<N>) -> f32 {
            let mut res = 0.0;
            for i in 0..N {
                res += self.inner[i] * other.inner[i];
            }
            res
        }

        pub fn append(mut self, value: f32) -> Vector<N + 1> {
            self.inner.push(value);
            unsafe { std::mem::transmute(self) }
        }

        pub fn split<const M: usize>(self) -> 
            (Vector<M>, Vector<N - M>) where M == N / 2 {
                let (a, b) = self.inner.split_at(M);
                unsafe {(Vector::from_vec(a.to_vec()), 
                         Vector::from_vec(b.to_vec()))}
            }
    }
    \end{minted}
    \caption{A vector implementation using generics over constants}
  \label{lst:vector_const_gen}
\end{listing}

Finally, the \inrust{split} method in \Fref{lst:vector_const_gen} is generic
over a constant \inrust{M} but is bounded to be equal to \inrust{N/2}, then
this parameter is used as the size of the return vectors acordingly. Given that
the returned vectors are created using \inrust{from_vec}, it is necessary to
use an \inrust{unsafe} block around the return.

This use case provides some valuable insights about the usage of generics over
constants. First, being able to use constants as generic type parameters does
not guarantee that each parameter coincides semantically with a particular
value (in this case, with the size of the vectors) and is the user's
responsibility to keep this semantic match. 

Even then, using this new kind of generics allows more expressiveness when
constants are involved. And simplifies the return type of functions with
verifications which can be done during compile time.

