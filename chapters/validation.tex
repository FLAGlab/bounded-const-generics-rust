% !TEX root = ../main.tex
\chapter{Validation}
\label{chap:validation}

Having discussed the design of our solution, this chapter addresses the
advantages and disadvantages of using this new kind of generics on an small
linear algebra module. This particular topic was chosen, given that numeric
computing makes heavy usage of statically sized linear data structures to
represent matrices and vectors.

Using Rust's \inrust{array} primitive type would be an unfair comparison,
given that in its current state, Rust is not able to abstract the size of the
arrays in a satisfactory manner. Instead we use the standard library
\inrust{Vec}, which is dynamically sized but it can be wrapped to restrict its
size.

Listings \ref{lst:vector_vanilla} and \ref{lst:vector_const_gen} contain two
implementations of a \inrust{Vector} structure with and without using generics
over constant values. Each implementation has three methods:

\begin{itemize}
    \item The \inrust{dot} method: Compute the dot product between the current
        vector and another vector given as parameter, it is required that both
        vectors have the same number of dimensions.  
    \item The \inrust{append} method: Consume the current vector and return a
        new vector which has an additional dimension with an element given as
        paramter.  
    \item The \inrust{split} \ method: Consume the current vector and return two
        vectors which contain the two halves of the current vector.  
\end{itemize}

The \inrust{Vector} type on \Fref{lst:vector_vanilla} has the size of the vector
stored inside the \inrust{inner} field and is accessed using
\inrust{self.inner.len()} as shown in the \inrust{size} method. Thus, it is
necessary to check if the \inrust{other} argument has the same size inside the
\inrust{dot} method and an \inrust{Option<f32>} is returned because there is no
certainty that the sizes of the vectors will coincide. The \inrust{append}
method on \Fref{lst:vector_vanilla} adds the new element to the \inrust{inner}
field and returns the same object. Finally, the \inrust{split} type uses the
method \inrust{split_at} of the \inrust{Vec} type and creates two new vectors
from the two halves of the \inrust{inner} field.

The first noticeable difference between the two implementations is the lack of
a \inrust{size} method in \Fref{lst:vector_const_gen}, this is because the size
of the vector is encoded in the \inrust{N} parameter of the type
\inrust{Vector<N>}. 

\begin{listing}[H]
    \begin{minted}{rust}
    pub struct Vector { inner: Vec<f32> }

    impl Vector {
        fn size(&self) -> usize {
            self.inner.len()
        }

        pub fn dot(&self, other: &Vector) -> Option<f32> {
            if self.size() == other.size() {
                let mut res = 0.0; 
                for i in 0..self.inner.len() {
                    res += self.inner[i] * other.inner[i];
                }
                Some(res)
            } else {
                None
            }
        }

        pub fn append(mut self, value: f32) -> Vector {
            self.inner.push(value);
            self
        }

        pub fn split(self) -> (Vector, Vector) {
            let (a, b) = self.inner.split_at(self.size() / 2);
            (Vector { inner: a.to_vec() }, 
             Vector { inner: b.to_vec() })
        }
    }
    \end{minted}
    \caption{A vector implementation without generics over constants}
  \label{lst:vector_vanilla}
\end{listing}

There is also a new method \inrust{from_vec_unchecked} which panics during
execution if the size of the \inrust{vec} argument is not \inrust{N}.  This is
necessary because, there is no way to do this verification during compilation,
as a consequence of the dynamic length of the \inrust{Vec} type. It is the
responsibility of the author of the \inrust{Vector<N>} type to use
\inrust{from_vec_unchecked} in cases when it will not panic. It would be
possible to use the \inrust{Option} or \inrust{Result} type, but this would add
an unnecessary overhead in this use case.

\begin{listing}[H]
    \begin{minted}{rust}
    pub struct Vector<const N: usize> { inner: Vec<f32> }

    impl<const N: usize> Vector<N> {
        fn from_vec_unchecked(inner: Vec<f32>) -> Vector<N> {
            if inner.len() == N {
                Vector { inner }
            } else {
                panic!("Creating a vector of incorrect size")
            }
        }

        pub fn dot(&self, other: &Vector<N>) -> f32 {
            let mut res = 0.0;
            for i in 0..N {
                res += self.inner[i] * other.inner[i];
            }
            res
        }

        pub fn append(mut self, value: f32) -> Vector<N + 1> {
            self.inner.push(value);
            Vector::from_vec_unchecked(self.inner)
        }

        pub fn split<const M: usize>(self) -> 
            (Vector<M>, Vector<N - M>) where M == N / 2 {
                let (a, b) = self.inner.split_at(M);
                (Vector::from_vec_unchecked(a.to_vec()),
                 Vector::from_vec_unchecked(b.to_vec()))
            }
    }
    \end{minted}
    \caption{A vector implementation using generics over constants}
  \label{lst:vector_const_gen}
\end{listing}

Another difference is the return type in the \inrust{dot} method. In
\Fref{lst:vector_vanilla} the method's return type is \inrust{Option<f32>},
this is because it is necessary to verify that both vectors have the same size
during runtime. In contrast, the \Fref{lst:vector_const_gen} implementation
does not require this verification inside the method because the type
compilation. The vector parameter \inrust{other} must have size \inrust{N} and,
as a consequence, calling this method with a vector of a different size would
result on a compile time error.

The differences on the \inrust{append} method reside in the return type. On
\Fref{lst:vector_const_gen}, the size of the return vector is increased by one
on its type parameter. However, in order to satisfy this, it is necessary to
use the \inrust{from_vec_unchecked} method to create a vector of size \inrust{N
+ 1}, this call will not panic given that after adding the new element, the
\inrust{inner} field will have length \inrust{N + 1}.

The \inrust{split} method in \Fref{lst:vector_const_gen} is generic
over a constant \inrust{M} but is bounded to be equal to \inrust{N/2}, then
this parameter is used as the size of the return vectors acordingly. Given that
the returned vectors have length \inrust{M} and \inrust{N - M} respectively,
using \inrust{from_vec_unchecked} will not panic.

This use case provides some valuable insights about the usage of generics over
constants. First, being able to use constants as generic type parameters does
not guarantee that each parameter coincides semantically with a particular
value (in this case, with the size of the vectors) and is the developer's
responsibility to keep this semantic match. 

As a consequence, developers must be careful when creating variables with
generic types over constant from variables without them. Even then, using this
new kind of generics allows more expressiveness when constants are involved 
and simplifies the return type of functions with verifications which can be
done during compile time.
