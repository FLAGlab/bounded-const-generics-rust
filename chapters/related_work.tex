% !TEX root = ../main.tex
\newcommand{\pdtn}[1] { \langle \text{#1} \rangle }

\chapter{Related Work}

\label{chap:related_work}

Abstracting over constants is a similar problem to abstracting over types as
with generic types. Thus, is no surprise that the solutions for the problems
stated in \fref{chap:motivation} are often related to parametric
polymorphism. In this chapter, mechanisms to parametrize code over constants are
discussed, both from a purely theoretical and an implementation point of view.

\section{Dependently typed programming languages}

Given the correspondence between dependent type systems and predicate logic,
early programming languages with dependent types were proof assistants, such as
Coq and Agda~\cite{agda}, both of which can be used to proof theorems
automatically.  However, such languages are not used outside the academic
community nor to write user or system oriented applications.

Haskell, being both a general purpose programming language and the de facto
language to explore and study new type systems in the academy, has provided the
foundation to study several formulations of dependent types in the form of new
languages, such as Agda, Idris \cite{idris} and Cayenne~\cite{cayenne}, or by
adding dependent types directly to the language. This last part is particularly
relevant to this work, given that Rust's nor Haskell's type system and compiler
were written to have dependent types.

C++ is the language that is most often compared to Rust because both languages
were designed to write performance oriented applications. Templates, which are
C++ mechanism to achieve parametric polymorphism, allow not only types as
parameters but also constant values~\cite{templates}. This idea is pretty
similar to the one being implemented for Rust in RFC 2000. Even then, manual
memory allocation makes C++ a language far from being memory safe. Where as Rust
is a completely memory safe language~\cite{ralf}.    

Both Haskell and C++ features are discussed in the remaining part of this
section.

\subsection{Dependent Haskell} 

The Glasgow Haskell compiler has certain features which makes Haskell almost a
dependently typed language. For example, datatype promotion allows algebraic
data types to be promoted as kinds, and each term of such types to be promoted
as a type itself. With these promotions, traditional polymorphism and type
families can be seen as a form of dependent types with some restrictions.

The addition of fully dependent types to Haskell's type system has been studied
deeply by \citet{gundry} and subsequently by \citet{eisenberg}. Both authors
add dependent types to Haskell by writing an intermediate language, the
\textit{evidence} language for Gundry's work, and Pico for Eisenberg's work.
These intermediate languages are compilation targets for the dependent version
of Haskell, and then the intermediate languages are compiled to standard
Haskell. Both Pico and the \textit{evidence} language are fairly similar to the
calculus of constructions, but have a different notion of type equivalence.

Refinement types, which are types with predicates, are a different theoretical
approach which provides to some extent capabilities similar to dependent types,
such as totality checking, program verification and proof automation.
LiquidHaskell is an extension to Haskell which allows refinement types. However,
it requires an SMT solver during compilation~\cite{liquidhaskell}. Thus,
following a similar approach in Rust would have several performance drawbacks.

\subsection{Templates in C++} 

C++ templates are widely used in the same way as generic types: They avoid code
repetition by parametrizing types and functions over other types. However,
templates are by themselves a powerful metaprogramming mechanism and not an
organic part of the language itself~\cite{template_metaprogramming}. This
extends the capability of templates beyond traditional generic types. For
example, templates allow non-type parameters in high contrast to generics in
languages such as Java.

There are some restriction with non-type parameters, they only can be constant
integer values, null pointers, or pointers to objects and
functions~\cite{templates}. Being limited to types parametrized over constant
values, C++ is not a dependently typed language, neither it has the same
capabilities as a dependently typed language. But being able to parametrize code
over constant values is a viable solution to some of the problems aforementioned
in \fref{chap:motivation}, in particular, having a similar system in
Rust would allow to implement traits for arrays of any size. Even then, memory
safety guarantees must be preserved.

\section{The state of Rust} 

\subsection{Rust's type system}

There are two main formalizations of Rust: 

\begin{itemize} 
    
    \item The first one done by \citet{reed}, provides a formalization of Rust
        called Patina, which describes Rust memory managment. A formal proof of
        Patina's soundness is given, based on the traiditional technique of
        progress and preservation. One of the main limitations of Patina is that
        it only includes the safe subset of Rust as a language. Any use of
        unsafe code cannot be represented properly using Patina.

    \item The second one developed by \citet{ralf}. Provides an small calculus
        known as $\lambda_{Rust}$ inspired deeply by Rust's MIR. In contrast
        with Patina, $\lambda_{Rust}$ can represent properly unsafe code and a
        extensible mechanism to proof the soundness of unsafe code is given. The
        soundness of $\lambda_{Rust}$ is proved using a semantic approach using
        Iris, a framework for higher-order concurrent separation logic, instead
        of using progress and preservation, which requires a fixed set of typing
        rules (this is a condition that unsafe Rust does not met).
        
\end{itemize}

\subsection{RFC 2000} 

\label{subsec:rfc2000}

Substantial changes to Rust must follow the RFC or request for comments process.
On September 14, 2017, an RFC regarding the addition of generics over constant
values was merged into the RFC
repository~\footnote{\url{https://github.com/rust-lang/rfcs/pull/2000}}. The
main motivation behind this RFC is to allow implementations of traits for all
array sizes~\footnote{\url{https://github.com/rust-lang/rfcs/blob/master/text/2000-const-generics.md}}. 

The proposed solution in this RFC is to add constant values to the possible
kinds of generic parameters (which currently are types and lifetimes as stated
in \fref{chap:preliminaries}), the proposed syntax to declare constant
parameters can be seen on \fref{lst:const_generics}. This RFC imposes
certain restrictions about when a constant parameter can be used, specifically
constant parameters cannot be used to initialize constant or static values,
functions and other types.

\begin{listing}[ht]
	\begin{minted}{rust}
    fn zeros<const N: usize>() -> [i32; N] {
        [0; N]
    }
	\end{minted}
    \caption{A generic function having a constant value as parameter}
  \label{lst:const_generics}
\end{listing}

This RFC takes structural equality as its notion of term equivalence, i. e. two
values are equivalent if all their inner fields are equal. It is also mentioned
that projections over constant parameters will not be considered equivalent even
if they represent the same value. As an example, the compiler will not be able
to typecheck the code in \fref{lst:uncheckable_code}, because the
projection \texttt{N + 1} appears on different places of this code's AST and is
treated as a different expression on each appearance.

\begin{listing}[ht]
	\begin{minted}{rust}
    fn push_zero<const N: usize>(arr: [i32, N]) -> [i32; N + 1] {
        let res: [i32; N + 1] = [0; N + 1];
        for i in 0..N {
            res[i] = arr[i];
        }
        res
    }
	\end{minted}
  \caption{After implementing RFC 2000, Rust's compiler will not be able to
  typecheck the function \texttt{zero}}
  \label{lst:uncheckable_code}
\end{listing}

Finally, this RFC states some unresolved questions about unification and well
formedness of constant expressions. In particular, it is stated that the absence
of a proper unification mechanism for projections over constant parameters will
result in poor user experience and that such a mechanism should be provided
before stabilizing this feature.

\subsection{The typenum crate}

The typenum~\footnote{\url{https://crates.io/crates/typenum/}} crate uses Rust's
types and traits to encode integers and operations between them. Each type-level
integer has associated functions to produce the corresponding value. For
example, the type \texttt{P1} represents the possitive integer \texttt{1} and
the function call \texttt{P1::as\_i32()} returns an \texttt{i32} value
containing \texttt{1}. 

Operations are represented by generic types whose parameters are the operation's
arguments. As an example, the type \texttt{Sum<A, B>} represents the output of
the addition between \texttt{A} and \texttt{B}. Thus, the type
\texttt{Sum<P1,P2>} is the same type as \texttt{P3}, the type-level integer
representing \texttt{3}. Currently there exist an implementation of arrays
generic over its size using typenum's
types~\footnote{\url{https://crates.io/crates/generic-array/}}.

Even though those crates allow to implement traits for arbitrary array sizes,
there are some ergonomic drawbacks. First, manipulating these generic arrays requires a
different syntax than the used by Rust's own arrays. Second, is impossible to
produce a type-level integer from an integer value, making type-level integers
hard to manipulate in certain situations.
