% !TEX root = ../main.tex
\chapter{Motivation}
One of the main objectives of Rust is to reduce the number of trade-offs for
the programmer. For example, fast code should not be unsafe, and language
abstractions should not have significant performance costs. Rust accomplishes
this by having a compiler that is able to reason more deeply about the code
than usual. Rust can reason about the lifetime of each variable to avoid
dangling pointers, or about mutability to avoid data races. 

However, some improvements could be done in the interaction of the language
with constant values. Not necessarily from a performance perspective, but also
from an ergonomic perspective. This chapter will explain some aspects of the
compilation and reasoning processes that could be improved.
\Fref{chap:solution} addresses the solution for the problems
posit hereinafter.

\label{chap:motivation}
\section{Trait implementations for arrays}
In Rust, the programmer has the capability to store data in the heap and the
stack. On the one hand, stack allocated values must have an static size (it
must be known at compile time). On the other hand, heap allocated values can
have dynamic size, but they can only be accessed using references. The
differences between vectors and arrays are a perfect example of this trade-off:
Vectors, being heap allocated, can grow or reduce in size during execution, but
having nested vectors causes performance issues, because access must be done
using nested references. Arrays, being stack allocated, have no performance
issues when nesting them, but their size must be fixed during execution.

These limitations create some ergonomic problems with arrays: Two arrays with
different sizes have different types, even if both store values of the same
type. Thus, trait implementations for array types must be done manually for
each possible size. Because of this, the standard library only implements
traits for array types up to a size of 32, even though arrays are a primitive
type of the language. In certain cases vectors are used instead of arrays.
Making the code unsuitable for applications running on embedded devices (which
may not allow heap allocation) or applications needing multidimensional
vectors.

An example of this limitation is shown in \fref{lst:trait_array}, where the
implementation of the trait \inrust{Volatile} cannot be easily extended to
other array sizes, even when such implementation does not use the size of the
array. 

\begin{listing}
	\begin{minted}{rust} 
    const N: usize = 3;

    trait Volatile {
        fn explode(&self);
    }

    impl<T> Volatile for [T; N] {
        fn explode(&self) {
            for _ in self {
                println!("Boom!")
            }
        }
    }

    fn main() {
        [0i32, 1, 2].explode(); 
        [0i32, 1].explode(); // stderr: no method named `explode` 
                             // found for type `[i32; 2]` in the
                             // current scope
    }
	\end{minted}
    \caption{Even though \inrust{Volatile} is implemented for \inrust{[T; 3]}, it is not for \inrust{[T; 2]}.}
  \label{lst:trait_array}
\end{listing}

\section{Static control flow and optimizations}
One of the recent efforts to improve Rust's performance is constant
propagation. If a constant can be resolved during compilation, the compiler
will propagate the constant through the code. For example, if the condition of
an if-else statement can be resolved during compilation, the compiler will
replace the statement with one just including the matching arm of the
statement.

An small illustration of this can be seen in \Fref{lst:static_control_flow},
where the array size \inrust{N} is a constant value known during compilation.
Given that the expression  \inrust{N > 0} can be evaluated during compilation,
the compiler will optimize the function, resulting in code similar to the code
shown in \Fref{lst:optimized}.

\begin{listing}
    \begin{minted}{rust}
    const N: usize = 0;

    fn head<T>(array: &[T; N]) -> Option<&T> {
        if N > 0 {
            Some(&array[0])
        } else {
            None
        }
    }
    \end{minted}
    \caption{This function must be optimized to improve performance.}
    \label{lst:static_control_flow}
\end{listing}

\begin{listing}
    \begin{minted}{rust}
    fn head<T>(array: &[T; 0]) -> Option<&T> {
        None
    }
    \end{minted}
    \caption{This function should be equivalent to the one in \fref{lst:static_control_flow} after constant propagation.}
    \label{lst:optimized}
\end{listing}

However, if a mechanism to abstract code over constant were added to Rust, the
constant propagation mechanism would not be able to optimize code with
unresolved constants. For example, if the value of \inrust{N} were not known in
\Fref{lst:static_control_flow}, it would not be possible to optimize the
\inrust{head} function even if always were used in empty arrays.
