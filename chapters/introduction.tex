% !TEX root = ../main.tex
\chapter{Introduction}
\label{chap:introduction}
The current ecosystem for systems programming has been based around having
powerful and fast programming languages, mainly C and C++, those languages
leave to the programmer the task of checking its code for stack overflow
errors, dangling pointers, and data races; forcing them to code inefficiently
to avoid such errors.

Modern programming languages try to solve this problem adding runtime
mechanisms to guarantee memory safety, such as garbage collection, making them
unsuitable for systems programming, where resource efficiency is vital.

The Rust programming language offers an alternative to C and C++ for systems
programming, using compile time mechanisms to guarantee memory safety even for
concurrent applications without sacrificing performance during execution. Even
then, the current state of ergonomics in Rust can be improved and several
efforts are being done by the Rust team and the community to improve the
language on these regards.

This work explores the weaknesses of Rust when code needs to be generalized
over constant values and proposes a solution to such problems by extending
Rust's type system to allow a basic form of dependent types which will only
allow constants values as indexes for types. To achieve such goal, the compiler
will be modified to allow constant values as parameters for generic types:

\begin{itemize}
    \item \Fref{chap:preliminaries} offers an introduction to the Rust
        programming language syntax, features and compiling internals relevant
        to this work. This chapter does not intend to be a full introduction to
        the language, just an attempt to set common terminology with the
        reader.
    \item \Fref{chap:motivation} contains a series of code examples of the
        problems commonly found when working with types depending over constant
        values. The two problems discussed on this chapter are the
        implementation of traits over the type of arrays (which is a dependent
        type) and the optimizations done by the Rust's compiler when handling
        static control flow. These problems are discussed not only in terms of
        their ergonomics and readability, but also in terms of its compilation
        and execution performance.
    \item \Fref{chap:related_work} discusses related work from a theoretical
        and practical perspective. It includes a section about the theory of
        dependently typed languages which is the theoretical foundation for
        generic types over constant values. The following section explores the
        different approaches to the implementation of dependent types done by
        two languages: C++ templates, and the work done on Haskell's type
        system by Gundry \cite{gundry} and Eisenberg \cite{eisenberg}. Studies
        about the Rust's type system are explored afterwards, mainly the work
        done by Jung et al \cite{ralf}. Finally, the current state of Rust
        regarding dependent types is discussed in the last chapter, which
        includes the 2000 Rust RFC
        \footnote{\url{https://github.com/rust-lang/rfcs/blob/master/text/2000-const-generics.md}}
        and the typenum crate.
        \footnote{\url{https://crates.io/crates/typenum}}
    \item \Fref{chap:solution} proposes a solution to the problems discussed in
        \fref{chap:motivation}, based on the idea of adding dependent types
        over constant values, known in the Rust community as generics over
        constant values, and a possible extension adding bounds to this new
        kind of generic parameters.
\end{itemize}
