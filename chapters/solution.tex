% !TEX root = ../main.tex
\chapter{Solution}
\label{chap:solution}

Even though the changes introduced to Rust on \fref{subsec:rfc2000} add the
necessary syntax to abstract code over constant expressions, Rust lacks of a
mechanism reason deeply about constant expressions, and as a consequence is not
possible to typecheck or provide any compile time guarantees over the safety of
using generics over constant values.  On this chapter, we introduce
\textsc{sire}, a symbolic evaluator for Rust, this provides a foundation to
treat function equality via an SMT solver. 

Symbolic execution evaluates a program in an abstract manner, taking symbols
representing each program's input, instead of concrete values, and then
executing the program propagating each symbol. With a symbolic representation
of a function is possible to reason deeply about its behaviour, in particular
is possible to decide if two functions evaluate to the same values for every
possible input or which input for a function would produce a particular result.

With this taken into account, is possible to extend Rust's compiler to
reason about generic types over constant values and provide type checking,
trait resolution and type inference for this new kind of generics.

The remaining part of this section explains the inner processes done by
\textsc{sir} and proposes a solution to the problems stated in
\fref{chap:motivation}.

\section{Symbolic execution of Rust programs}
\label{sec:symbolic_execution}

\textsc{sire} is a symbolic executor for the \textsc{mir} of Rust, which is
able to interact with the Rust's compiler. After the compiler generates the
\textsc{mir} for a function, \textsc{sire} takes this representation and
evaluates it into an small symbolic intermediate representation or \textsc{sir}
for short.

Currently \textsc{sire} can only evaluate an small subset of all possible
functions. Specifically, it can evaluate functions without mutable arguments nor
mutable references as arguments containing the following subset of \textsc{mir}:

\begin{itemize}
    \item Statements: \inrust{Assign}, \inrust{StorageLive} and \inrust{StorageDead}.
    \item Terminators: \inrust{Goto}, \inrust{Return}, \inrust{Call} and \inrust{SwitchInt}.
    \item Rvalues: \inrust{BinaryOp}, \inrust{Ref} and \inrust{Use}.
    \item Operands: \inrust{Move}, \inrust{Copy} and \inrust{Constant}.
\end{itemize}

In addition, \textsc{sire} only supports integer and boolean types. Support for
structures, tuples, enumerations and arrays is planned in the future. Floating
point types are not supported given that they lack of the structural equality
property. \footnote{This was stated on
\href{https://github.com/rust-lang/rfcs/blob/master/text/1445-restrict-constants-in-patterns.md}{RFC-1445}.}

\textsc{sire} has an store to read and write symbolic expressions during
evaluation. The symbolic execution of a function starts by allocating into its
store each of the function's arguments as symbols, then each \textsc{mir}
expression is evaluated into a \textsc{sir} expression. When the return
statement of the \textsc{mir} of the function is reached, \textsc{sire} takes
from its store the symbolic expression corresponding to the return value of the
function and returns it. Providing a symbolic representation of the function. 


\subsection{Symbolic intermediate representation}

\textsc{sir} is a relatively simple language. Functions are the main construct
of \textsc{sir} and function arguments are numbered following the same
convention as \textsc{mir} where the zeroth argument is the return place.  The
grammar for \textsc{sir} can be found on \fref{lst:sir_grammar}.

The body of each function is composed of expressions which can be function
applications, binary operations, switch statements or pure values. 

There are only three kinds of values: Function arguments, constants and
function names. 

Finally, switch statements are composed by the value to be compared, the
possible values that it can take, and the result for each possible value (there
must be always a default result). 

The evaluation done by \textsc{sire} for each one of the \textsc{mir}
expressions mentioned on section \fref{sec:symbolic_execution} is explained
on table \fref{tab:sire_table}.

\begin{table}[ht]
    \centering
    \begin{tabular}{ | c | c | c | }
    \end{tabular}
    \caption{Evaluation of each \textsc{mir} expression, showing the action
    done over the store by \textsc{sire}, and the output \textsc{sir}
    expression. }
  \label{tab:sire_table}
\end{table}


As an example, when executing the code on \fref{lst:rust_sire_example}, the
\textsc{mir} and \textsc{sir} generated can be found on
\fref{lst:mir_sire_example} and \fref{lst:sir_sire_example}.

\begin{listing}[ht]
    \begin{minted}{ebnf}
    defun   = '(defun' name {ty} expr ')';
    expr    = value | '('expr {expr}')' | '('op expr expr')' | 
             '(switch' expr {'('expr '->' expr')'} defcase);
    value   = '_'num | '(const' ty num')' | name;
    defcase = '(else ->' expr')';
    ty      = '(int' num')' | '(uint' num')' | 'bool';
    num     = ? a positive integer ?;
    name    = ? an string denoting the name of a function ?;
    op      = ? a binary operator ?;
    \end{minted}
    \caption{\textsc{sir}'s grammar in EBNF}
  \label{lst:sir_grammar}
\end{listing}

\begin{listing}[ht]
    \begin{minted}{rust}
    fn distance(x: i32, y: i32) -> i32 {
        if x > y {
           x - y
        } else {
           y - x
        }
    }
    \end{minted}
    \caption{A simple Rust function to be evaluated using \textsc{sire}}
  \label{lst:rust_sire_example}
\end{listing}

\begin{figure}[ht]
    \centering
    \includegraphics[height=12cm]{images/distance.pdf}
    \caption{The \textsc{mir} of the \inrust{distance} function on \fref{lst:rust_sire_example}}
  \label{lst:mir_sire_example}
\end{figure}

\begin{listing}[ht]
    \begin{minted}{lisp}
    (defun distance (int 32) (int 32) (int 32) 
        (switch (> _1 _2) 
            ((const bool 0) -> (- _2 _1)) 
             (else -> (- _1 _2))))
    \end{minted}
    \caption{The \textsc{sir} of the \inrust{distance} function on \fref{lst:rust_sire_example}}
  \label{lst:sir_sire_example}
\end{listing}

\begin{listing}[ht]
    \begin{minted}{lisp}
    (define-fun distance 
     ((x1 (_ BitVec 32)) (x2 (_ BitVec 32))) (_ BitVec 32) 
     (ite (bvsgt x1 x2) (bvsub x1 x2) (bvsub x2 x1)))
    \end{minted}
    \caption{The \textsc{smt-lib} snippet for the \textsc{sir} of the \inrust{distance} function on \fref{lst:rust_sire_example}}
  \label{lst:smt_sire_example}
\end{listing}


\subsection{Equality of symbolic functions}

Two \textsc{sir} functions are considered equal if they evaluate to the same
expression for every possible argument. For simple arithmetic expressions, this
could be achieved via E-unification. However, \textsc{sir} functions contain
control flow operations and recursive calls, in this case a theorem prover such
as Z3 is up to the task. \textsc{sire} can transform every \textsc{sir}
function into an small snippet in the \textsc{smt-lib} format to use it in an
SMT solver.

Integer types on \textsc{sir} are transformed into \textsc{smt-lib} bit vectors
of the corresponding length and the signed or unsigned arithmetic operations
are transformed accordingly.  Switch expressions are transformed into nested
conditionals. An example of such snippets can be found on
\fref{lst:smt_sire_example}. 

To decide if two functions are equal, is enough to write an small
\textsc{smt-lib} snippet checking if the two functions are equal in their whole
range. On \fref{lst:func_equality}, the \inrust{distance} function is compared
with an alternative implementation found on \fref{lst:alt_distance}. 

\begin{listing}[ht]
    \begin{minted}{rust}
    fn alt_dist(x: i32, y: i32) -> i32 {
        let sign: i32; 
        if x > y {
           sign = 1;
        } else {
           sign = -1;
        }
        sign * (x - y)
    }
    \end{minted}
    \caption{An alternative implementation of the \inrust{distance} function on \fref{lst:rust_sire_example}}
  \label{lst:alt_distance}
\end{listing}

\begin{listing}[ht]
    \begin{minted}{lisp}
    ;; Definition of distance provided by sire
    (define-fun distance 
     ((x1 (_ BitVec 32)) (x2 (_ BitVec 32))) (_ BitVec 32) 
     (ite (bvsgt x1 x2) (bvsub x1 x2) (bvsub x2 x1)))

    ;; Definition of alt_dist provided by sire
    (define-fun alt_dist 
     ((x1 (_ BitVec 32)) (x2 (_ BitVec 32))) (_ BitVec 32) 
     (ite (bvsgt x1 x2) 
      (bvmul (_ bv1 32) (bvsub x1 x2)) 
      (bvmul (_ bv4294967295 32) (bvsub x1 x2))))
    
    ;; Assert that the functions are equal
    (assert (forall ((x1 (_ BitVec 32)) (x2 (_ BitVec 32))) 
     (= (distance x1 x2) (alt_dist x1 x2)))) 

    ;; Check if the assertions can be satisfied
    (check-sat) ; sat
    \end{minted}
    \caption{equality check between the \inrust{distance} and \inrust{alt\_dist} functions}
  \label{lst:func_equality}
\end{listing}

\section{Typechecking of generics over constants}

\section{Generic traits over constants}

\begin{listing}[ht]
	\begin{minted}{rust}
    impl<'a, 'b, A: Sized, B, const N: usize> PartialEq<[B; N]> 
    for [A; N] where A: PartialEq<B> {
        fn eq(&self, other: &[B; N]) -> bool {
            self[..] == other[..]
        }
        fn ne(&self, other: &[B; N]) -> bool {
            self[..] != other[..]
        }
    }
	\end{minted}
    \caption{Implementing the \inrust{PartialEq} trait for all array sizes}
  \label{lst:trait_const_generics}
\end{listing}

\section{Bounds for generics over constants}

\begin{listing}[ht]
	\begin{minted}{rust}
    fn head<T, const N: usize>(array: [T; N]) -> T
    where {N > 0} {
        array[0]
    }
    \end{minted}
    \caption{Type-safe access to the first element of an array without using
    \inrust{Option<T>}}
  \label{lst:head_const_generics}
\end{listing}
\section{Validation}

In order to show that such extensions to Rust improve the ergonomy of the
language, an small library for numerical computing with and without our extended
syntax will be written. For both libraries, their cyclomatic complexity will be
computed and compared against each other. Given that these extensions to the
language remove some uses of control flow, a smaller complexity is expected.
