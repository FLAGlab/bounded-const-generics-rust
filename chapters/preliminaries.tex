% !TEX root = ../main.tex
\chapter{Preliminaries}

\label{chap:preliminaries}

Rust is a systems programming language focused on speed, memory safety and
concurrency. \footnote{\url{https://www.rust-lang.org/}}  It intends to offer
both performance similar to C++ and memory safety similar to Haskell. Rust is a
compiled language combining imperative and functional programming features. The
focus on performance makes Rust an ideal candidate for writing operative
systems, databases, compilers, and other high-performance software without
worrying about manual memory allocation. However, Rust's safety and high-level
abstractions has encouraged its usage in backend and even frontend development. 

Rust started as a personal project of Graydon Hoare during 2006, who wanted to
write a memory-safe, suited for concurrency and compiled language. After three
years, Mozilla endorsed the project and Rust was announced to the public during
2010. Since then Rust's development has been completely open to the community.
In 2013, Hoare stepped down as the technical leader of the project and a core
team for the project was formally established. \cite{steve_acm} Version 1.0 of
the language was released in May, 2015, following a six weeks release cycle
enforcing semantic versioning, the current version of the Rust compiler
\footnote{The current version is version 1.30} is completely backwards
compatible with version 1.0.

Currently, several companies have written in Rust part of their core
applications (including Mozilla, which is working on its next generation web
engine: \href{https://servo.org/}{Servo}) as can be seen in
\href{https://www.rust-lang.org/en-US/friends.html}{Rust Friends website}.

In the remaining of this section contains a quick tour of the Rust programming
language main features. Readers more experienced with Rust can skip this
explanation.

\section{Mutability}

Variable declarations are done using the \texttt{let} keyword. However, all
variables are immutable by default as shown in \fref{lst:immutable}.

\begin{listing}[ht]
	\begin{minted}{rust}
    fn main() {
        let x = 5;
        x += 1; // stderr: cannot assign twice to immutable variable `x`
        println!("{}", x);
    }
	\end{minted}
  \caption{Trying to modify an immutable value will result in a compilation error}
  \label{lst:immutable}
\end{listing}

Mutability is allowed using the \texttt{mut} keyword as shown in the declaration
of the variable $x$ in \fref{lst:mutable}. The explicitness of mutability not
only allows the compiler to do optimizations, it is also useful to the
programmer, if a variable is not declared explicitly as mutable, its value will
not change during its whole lifetime.

\begin{listing}[ht]
	\begin{minted}{rust}
    fn main() {
        let mut x = 5;
        x += 1;
        println!("{}", x); // stdout: 6 
    }
	\end{minted}
  \caption{Mutability is allowed but it must be explicit}
  \label{lst:mutable}
\end{listing}

\section{Memory management}
In Rust, memory safety checks are done during compilation, avoiding the need for
a garbage collector or manual memory management. The Rust compiler can reason
about memory usage via three concepts: Ownership, borrowing, and lifetimes.

\subsection{Ownership}
The semantics of value assignation in Rust differs from the semantics of C/C++
or Java. When assigning a value to a variable the state of the program change as
usual, but the variable becomes the new \textit{owner} of such value.
\cite{ownership_types} This means that the value will be dropped when its owner
goes out of scope. Each value can only have a single owner at the same time.
Meaning that for each value stored in memory there is exactly a single variable
owning it. 

When a value is reassigned to another variable, ownership is transferred i.e,
the former owner loses the ownership and it cannot be used again unless a new
value is assigned to it. As a consequence, when a variable is used as a function
argument, the variable loses ownership of its value, and the variable
representing the argument of the function becomes the new owner. This behavior
can be seen in \fref{lst:ownership}.

\begin{listing}[ht]
	\begin{minted}{rust}
    fn exclamate(z: String) {
        println!("{}!", z);
    }

    fn main() {
        let x = String::from("Hello, world");
        let mut y = x; // `y` is the new owner, `x` is invalid.
        exclamate(y); // `z`is the new owner, `y` is invalid.
        println!("{}", x); // stderr: use of moved value: `x`
        println!("{}", y); // stderr: use of moved value: `y`
    }
	\end{minted}
  \caption{Ownership transfer}
  \label{lst:ownership}
\end{listing}

The ownership restriction has two advantages: First, it is impossible to have a
value stored in memory without a variable in the current scope assigned to it,
avoiding some memory leaks. Second, is impossible to modify a single value from
several different threads, avoiding data races. Nevertheless, ownership forces
the programmer to write code akin to continuation-passing style, which is
error-prone and difficult to read. The \textit{borrowing} concept (described in
the next subsection) solves this issue.

\subsection{Borrowing}

Rust is a language with references. When a reference to a value is created, such
value is being borrowed (but ownership is not transferred). There are two kind
of references in Rust: immutable references denoted by \texttt{\&T} and mutable
references \texttt{\&mut T}. Immutable references allow "read-only" access,
independently of the referenced variable mutability. Mutable references allow
"read and write" access, but they only reference mutable variables. Examples of
both kinds of references can be found in \fref{lst:immutable_ref} and
\fref{lst:mutable_ref} respectively.

\begin{listing}[ht]
	\begin{minted}{rust}
    fn exclamate(z: &String) {
        println!("{}!", z);
    }

    fn main() {
        let x = String::from("Hello, world");
        exclamate(&x); // `z` is borrowing the value owned by `x`.
                       // stdout: Hello, world! 
        println!("{}", x); // stdout: Hello, world
    }
	\end{minted}
  \caption{References avoid the need for ownership transfer}
  \label{lst:immutable_ref}
\end{listing}

There are three rules about borrowing enforced by the compiler:
\begin{itemize}
    \item Several immutable references to a value can exist at a given time.
    \item There must be at most one mutable reference to a value at a given time.
    \item The first two scenarios are exclusive, only one of them can happen at the same time.
\end{itemize}

\begin{listing}[ht]
	\begin{minted}{rust}
    fn exclamate_in_place(z: &mut String) {
        z += &"!";    
    }

    fn main() {
        let mut x = String::from("Hello, world");
        exclamate(&mut x); // `z`is borrowing the value owned by `x`.
        println!("{}", x); // stdout: Hello, world!
    }
	\end{minted}
  \caption{Mutable references allow mutation of the borrowed value}
  \label{lst:mutable_ref}
\end{listing}

In other words, is possible to do just one of the following at the same time:
Have several readers or, have a single writer. This prevents the mutation of
shared state and, as a consequence, prevents data races in concurrent
applications. There are certain scenarios where the borrowing rules are not
flexible enough to allow certain kind of behaviors, such as locks for example,
in those cases is possible to use types with internal mutability, the
\texttt{Mutex} type is an example of this.

\subsection{Lifetimes}

Adding references to a language makes possible to have dangling references. When
a value is dropped, all its references become dangling references, no longer
pointing to a valid memory location. To solve this memory issue, Rust introduces
the concept of lifetimes, each value has a lifetime which starts when the value
is allocated and ends when the value is dropped. As a consequence, a value
lifetime ends when its owner goes out of scope. The Rust compiler has a "borrow
checker", which compares scopes to check that no reference outlives the value
being referenced. If this is not the case, a compilation error occurs.

In principle, every reference needs a lifetime annotation. However, this is
avoided by a process known as lifetime elision, where lifetimes are inferred
automatically by the compiler. Even then, in certain cases the programmer might
need to add such annotations. However, this will be discussed in the generics
section.

\section{Algebraic data types}

Rust has both sum and product algebraic data types in the form of structures and
enumerations respectively.

Structures are named sum types, where an struct type has a fixed set of named
fields. When declaring a new value of an struct type, all its fields must be
given as shown in \fref{lst:struct}. Enumerations are named product types, where
an enum type has a fixed set of named variants. When declaring a new value of an
enum type, one single variant must be chosen as shown in \fref{lst:enum}.

\begin{listing}[ht]
	\begin{minted}{rust}
    struct Pixel {
        red: u8,
        green: u8,
        blue: u8,
    }

    fn main() {
        let yel = Pixel {
            red: 255,
            green: 255,
            blue: 0
        };

        println!("({}, {} ,{})", yel.red, yel.green, yel.blue);
    }
    \end{minted}
  \caption{A structure representing the color of a pixel}
  \label{lst:struct}
\end{listing}

It is also possible to add associated functions to any type using the
\texttt{impl} keyword in an object oriented programming style. Each associated
function could or could not use an instance of a given type. This is similar,
for example, to the way instance and static methods are defined in Java.

\begin{listing}[ht]
	\begin{minted}{rust}
    enum Color {
        RGB(u8, u8, u8),
        CMYK(u8, u8, u8, u8),
    }

    fn main() {
        let yel = Color::RGB(255,255,0);
        let other_yel = Color::CMYK(0, 0, 90, 0);
        ...
    }
    \end{minted}
  \caption{An enumeration representing colors in different color systems}
  \label{lst:enum}
\end{listing}

\section{Control flow}

Control flow in Rust is realized using the common \texttt{if/else} conditional
and \texttt{while} loop statements, as in other programming languages.
\texttt{for} loops are iterator based, where the variable to be iterated must
implement the \texttt{Iterator} trait.

Rust has pattern matching capabilities thanks to its functional heritage,
pattern matching in Rust can be used to match specific values or to destructure
tuples, arrays, enumerations or structures. There are three statements to do
pattern matching: \texttt{match}, \texttt{if let} and \texttt{while let}.
Examples for these statemens can be found in listings \ref{lst:match},
\ref{lst:if_let} and \ref{lst:while_let} respectively.

\begin{listing}[ht]
	\begin{minted}{rust}
    enum List {
        Empty,
        Cons(i32, Box<List>)
    }

    impl List {
        fn length (&self) -> usize {
            match self {
                List::Empty => 0,
                List::Cons(_, cdr) => 1 + cdr.length()
            }
        } 
    }
    \end{minted}
    \caption{Computing the length of a list using the \texttt{match} statement}
  \label{lst:match}
\end{listing}

\begin{listing}[ht]
	\begin{minted}{rust}
    impl List {
        fn length (&self) -> usize {
            if let List::Cons(_, cdr) = self {
                1 + cdr.length()
            } else {
                0
            }
        } 
    }
    \end{minted}
    \caption{An alternative implementation of \texttt{length} in \fref{lst:match} using the \texttt{if let} statement}
  \label{lst:if_let}
\end{listing}

\begin{listing}[ht]
	\begin{minted}{rust}
    impl List {
        fn length (&self) -> usize {
            let mut counter = 0;
            let mut cons = self;
            while let List::Cons(_, cdr) = cons {
                counter += 1;
                cons = cdr;
            }
            counter
        } 
    }
    \end{minted}
    \caption{An alternative implementation of \texttt{length} in \fref{lst:match}
    using the \texttt{while let} statement}
  \label{lst:while_let}
\end{listing}

\section{Generics}

From an external perspective, generic types in Rust are quite similar to the
generic types in Java --that is, every type can have generic type parameters in
order to reduce code duplication and such parameters are specified between
angled brackets. However, there are three main differences between the two
implementations:

\begin{itemize}
    \item Rust allows lifetimes as generic parameters. These are used when the
        borrow checker cannot infer a proper lifetime for a variable, and thus
        explicit lifetime annotations are required. An example of this can be
        found in \fref{lst:gen_lifetimes}.
  
        \begin{listing}[ht]
            \begin{minted}{rust}
            fn longest<'a>(x: &'a List, y: &'a List) -> &'a List {
                if x.length() > y.length() {
                    x
                } else {
                    y
                }
            }
            \end{minted}
            \caption{Returning the longest list, lifetime annotations are required
            because the Rust compiler cannot decide if the return value will outlive
        \texttt{x} and \texttt{y}.}
          \label{lst:gen_lifetimes}
        \end{listing}
    
    \item In Java, generic parameters can be bounded by forcing them to be
        instances of a class or interface. In Rust, which is not an object
        oriented language, bounds are done over traits instead, as in
        \fref{lst:bounds}.
  
        \begin{listing}[ht]
            \begin{minted}{rust}
            fn increase<T: AddAssign + Copy>(vec: &mut Vec<T>, inc: T) {
                for elem in vec {
                    *elem += inc;
                }
            }
            \end{minted}
            \caption{A function which increments the elements of a vector by a fixed
                amount, this can only be done if the type of the elements \texttt{T}
                implements both the \texttt{AddAssign} and \texttt{Copy} traits.}
          \label{lst:bounds}
        \end{listing}

    \item Internally, generics in Java are implemented doing type erasure, where
        each generic parameter bounds are checked, and then the Java compiler
        forgets about the type of such paremeter and uses dynamic dispatch over
        the type \texttt{Object}. On the other hand, Rust uses monomorphization,
        where the compiler builds a new type specialized for each instance of a
        generic parameter making all the dispatch completely
        static.\footnote{Rust allows type erasure via trait objects. However,
        this goes is beyond the scope of this work.}
\end{itemize}

\section{Traits}

Traits are Rust's mechanism to allow ad-hoc polymorphism, they allow to extend
the behavior of a type requiring that the type implements a set of methods
defined by the trait. The main difference between using traits instead of
generic functions consist in the possibility to use concrete properties of an
specific type when implementing the trait. \cite{traits}

\begin{listing}[ht]
	\begin{minted}{rust}
    trait Volatile {
        fn explode(&self);
    }

    impl Volatile for i32 {
        fn explode(&self) {
            for _ in 0..*self {
                println!("Boom!");
            }
        }
    }
    \end{minted}
  \caption{Implementation of an user defined trait for a foreign type}
  \label{lst:trait_foreign_impl}
\end{listing}

User defined traits can be implemented for any type, in contrast to Java
interfaces where the implementations are restricted to the types declared in the
same package as the interface, as in \fref{lst:trait_foreign_impl}. On the other
hand, the user can implement a foreign trait for its own types, as in
\fref{lst:foreign_trait_impl}. However, the user can not implement foreign traits
for foreign types as shown in \fref{lst:foreign_trait_foreign_impl}.

\begin{listing}[ht]
	\begin{minted}{rust}
    use std::ops::Add;

    struct Rational {
        a: i32,
        b: i32,
    }

    impl Add for Rational {
        type Output = Rational;
        
        fn add(self, other: Rational) -> Rational {
            Rational {
                a: self.a * other.b + other.a * self.b,
                b: self.b * other.b
            }
        }
    }
    \end{minted}
  \caption{Implementation of a foreign trait for an user defined type}
  \label{lst:foreign_trait_impl}
\end{listing}

Traits can have associated types, such types can be used in the signature of the
trait associated functions to allow more expressiveness, as an example, in
\fref{lst:trait_foreign_impl}, \texttt{Output} is an associated type of
\texttt{Add} and it can be used as the return type of the \texttt{add} method,
allowing the addition of two variables of the same type, return a different
type. It is also possible to parametrize traits using types and lifetimes, i.e.,
generic traits.

\begin{listing}[ht]
	\begin{minted}{rust}
    use std::ops::Add;

    impl Add for bool { // stderr: only traits defined in the
                        // current crate can be implemented 
                        // for arbitrary types.
        type Output = bool;
        
        fn add(self, other: bool) -> bool {
            self || other
        }
    }
    \end{minted}
  \caption{Implementation a foreign trait for a foreign type results in a compilation error}
  \label{lst:foreign_trait_foreign_impl}
\end{listing}

Traits are used for operator overloading, e.g., the types that can be operated
with \texttt{+} must implement the \texttt{Add} trait of the standard library,
\fref{lst:foreign_trait_impl} is an example of this. 

Thread safety is also handled using traits. Types which are thread-safe to send
must implement the \texttt{Send} trait and types which are thread-safe to share
(using references) must implement the \texttt{Sync} trait. Both traits are empty
, i.e., they do not request any function to be implemented, but are unsafe
traits because the compiler can not guarantee the safety of sending or sharing
values of a certain type.

\section{Error handling}

It is common to use exceptions in imperative languages to represent errors.
However, having a functional influence, Rust has two kind of errors: 

\begin{itemize}
    \item Recoverable errors, where the program execution is not interrupted and
        is possible to handle the error.
    \item Unrecoverable errors, where the program execution stops and memory is
        cleaned.
\end{itemize}

Recoverable errors are written by returning a variable of type \texttt{Result<T,
E>}, which is an enumeration with two variants: the \texttt{Ok<T>}
variant,containing the result of a successful operation, and the \texttt{Err<E>}
variant, containing the failure reason of a failed operation. If a function
returns a value of type \texttt{Result<T, E>} the programmer must explicitly
handle both variants, making error handling explicit all the time. Error
propagation can be done using the \texttt{?} operator, which does an early
return if the expression returns an \texttt{Err<E>}. An example of this can be
seen on \fref{lst:recoverable_error}.

\begin{listing}[ht]
	\begin{minted}{rust}
    use std::io;
    use std::io::Read;
    use std::fs::File;

    fn read_file_to_string(path: &str) -> Result<String, io::Error> {
        let mut string = String::new();
        // if opening the file fails, the error is returned
        let mut file = File::open(path)?;
        // if reading the file fails, the error is returned
        file.read_to_string(&mut string)?;
        // if nothing fails, the string is returned
        Ok(string)
    }
    \end{minted}
  \caption{A function returning a recoverable error, doing error propagation}
  \label{lst:recoverable_error}
\end{listing}

Unrecoverable errors are written using the macro \texttt{panic!}, having as
argument a string; the panic reason. Unrecoverable errors are reserved for cases
when execution would cause undefined behavior or when it does not make sense to
keep executing the program after the error was reached. Because of this,
unrecoverable errors are most commonly used in code handling memory or other
low-level operations, such as vector insertion,
\footnote{\url{https://doc.rust-lang.org/std/vec/struct.Vec.html\#method.push}}
or when the logic of the application dictates the application must crash, as
seen in \fref{lst:unrecoverable_error}.


\begin{listing}[ht]
	\begin{minted}{rust}
    fn main() {
        match read_file_to_string("./config") {
            Ok(string) => { ... }
            Err(_) => panic!("Cannot run without a config file!")
        }
    }
    \end{minted}
  \caption{A function panicking after a critical error}
  \label{lst:unrecoverable_error}
\end{listing}

\section{Macros}

Macros are the mechanism used in Rust to allow metaprogramming. Rust's
capabilities on this matter are limited, but macros are widely used given the
static nature of the language. Currently Rust have two kind of macros:
procedural and declarative, given that only declarative macros are relevant to
the problems addressed by this work. Procedural macros will not be discussed
here.

Declarative macros, which are called using the syntax \texttt{macro\_name!},
allow to manipulate Rust code in a pattern matched way. These macros are
declared using the macro \texttt{macro\_rules!}, which take the name of the
macro and a series of code patterns like a \texttt{match} statement. 

Most of the time, declarative macros are used as a replacement for functions
with several arguments, given that Rust only admits functions with a fixed
number of arguments. For example, \texttt{vec!} and \texttt{println!} are macros
because they may receive a variable number of elements: \texttt{vec!(1)},
\texttt{vec!(1, 2)},  \texttt{vec!(1,2,3)} and so on.


\begin{listing}[ht]
	\begin{minted}{rust}
macro_rules! list {
    ($head:tt $(, $tail:tt)*) => {
        List::Cons(
            $head, 
            Box::new(list!($($tail),*))
        )
    };
    () => {List::Empty};
}

fn main() {
    let x = list!(1,2,3);
    println!("{}", x.length()); // stdout: 3
}
    \end{minted}
  \caption{A macro based constructor for lists}
  \label{lst:declarative_macro}
\end{listing}

\section{Intermediate representations}

Rust code is not compiled directly into machine code, instead is compiled into a
series of intermediate representations. On early versions, Rust code suffered a
desugaring process (this representation is known as the high-level intermediate
representation or HIR) before being compiled into the LLVM intermediate
representation, and finally it is compiled to machine code.
\footnote{\url{https://rust-lang-nursery.github.io/rustc-guide/codegen.html}}

\begin{figure}[ht]
  \centering
  % \includegraphics[height=1.5cm]{images/original_pipeline.pdf}
  \caption{Rust's compilation pipeline before 2016.}
\end{figure}

During 2016, Rust added the mid-level intermediate representation or MIR to its
compilation pipeline. This new representation improved the borrow checking and
optimization processes, allowing for both faster and more readable code.
\footnote{\url{https://blog.rust-lang.org/2016/04/19/MIR.html}}

On this same year, miri, an intepreter for the MIR was written. This interpreter
can execute code written in the MIR directly instead of compiling it to machine
code. \footnote{\url{https://solson.me/miri-report.pdf}} However, Rust is still
a compiled language, miri is not used currently by the Rust project as a
"virtual machine". Instead, miri is used to evaluate constant expressions during
compilation.

\begin{figure}[ht]
  \centering
  % \includegraphics[height=1.5cm]{images/current_pipeline.pdf}
  \caption{Rust's compilation pipeline after 2016.}
\end{figure}
