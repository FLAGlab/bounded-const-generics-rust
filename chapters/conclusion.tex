% !TEX root = ../main.tex
\chapter{Conclusion and Future work}
\label{chap:conclusion}

\section{Conclusion}

The objective of this thesis was to provide Rust with generics over constants
as a means to facilitate the language's capabilities to write generic code over
constants.

In this regard, it is certain that adding constant expressions as a kind of
parameter over generic types has improved the expressiveness and ergonomy of
Rust when dealing with constant abstraction. In particular, these additions
make the array types first class entities of the language, allowing to
implement traits over all its sizes and providing a safe mechanism to
manipulate them with static guarantees. These changes also allow for a more
natural definition of user defined types which have information that will not
change during execution and thus can be encoded in constant values as generic
parameters.

It is expected that following this design would allow for a more performant
execution of functions dealing with statically verifiable conditions. However,
it is important to remark that this design would increase compilation time
given the complexity of verifying if a set of statements is satisfiable.

Some prospects of future work are given in the rest of this chapter.


\section{Future work}

Even though \textsc{sire} is able to interpret a subset of the Rust's
\textsc{mir}, it is important to extend the coverage of \textsc{mir}
expressions. In particular, we consider of utmost importance adding support for
loops, panicking functions and algebraic data types:

\begin{itemize}

    \item Loops are the most common way of writing repetitive tasks in Rust
        even though recursion is supported. Given that loops are represented in
        \textsc{mir} as blocks with their terminators looping between them,
        this would require promoting loops to functions to transform it to a
        recursive representation suitable for the \textsc{smt-lib} format.         

    \item One of the biggest concerns of using any kind of generics is allowing
        programs with generic types that would produce unexpected behaviour after
        monomorphization. Checking on which cases a function panics, would allow
        the compiler to avoid monomorphization of generic types on those cases, or
        to at least warn the user about them.

    \item On the other hand, algebraic data types are needed to implement
        interesting structures such as bounded integers to improve the ergonomy
        of type safe access of arrays and to allow the usage of constants with
        types \inrust{Option<T>} and \inrust{Result<T, E>} as type parameters.
        To fully support algebraic data types, it is necessary to support
        \textsc{mir} projections to represent field access in structures.

\end{itemize}

Even though using an SMT solver makes our solution simpler in terms of
implementation, it adds an additional dependency if these changes were to be
added to the compiler. It is also true that an SMT solver is far from the
notion of zero-cost abstraction, because it would be more performant to write a
logic system to deal only with Rust's characteristics. Thus, we consider
necessary to provide an alternative verification mechanism, one option would be
to extend the chalk\footnote{\url{https://github.com/rust-lang/chalk}} project
to deal with these verifications.

As a final prospect, integrating this project with the compiler codebase would
simplify maintenance of the new features exposed on this work. This codebase
would fit under the constant evaluation modules of the compilers code, given
its closeness to \textsc{miri}.
